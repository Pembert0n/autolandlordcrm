\documentclass[a4paper,onecolumn,draft,pdftex]{report}
\usepackage{ngerman}
\usepackage[utf8x]{inputenc}

\begin{document}
\author{n.n. \\
Universit"at Münster}
\title{\LaTeX{} Testdokument}
\date{November 2005}
\setcounter{secnumdepth}{3}
\maketitle


\begin{abstract}
Hier steht die Zusammenfassung des Textes. LaTeX ist ein m\"achtiges und flexibles Satzsystem, das sich besonders f\"ur
wissenschaftliche und technische Publikationen eignet.
\end{abstract}


\chapter{Anfangskapitel}
\section{Abschnitt1}
Dies ist der erste Abschnitt des ersten Kapitels. Hierbei soll gezeigt werden,
wie man einfach auf zweispaltigen Satz umschalten kann.\\[4cm]
Nach einem Zeilenumbruch von 5 cm geht es einfach weiter.
\subsection{Unterabschnitt}

LaTeX ist ein m\"achtiges und flexibles Satzsystem, das sich besonders f\"ur
wissenschaftliche und technische Publikationen eignet. Autoren k\"onnen
aus einer Vielzahl von fertigen Layouts ausw\"ahlen und diese eigenen 
Vorstellungen anpassen. 
\subsubsection{Noch ein Unterabschnitt}

Mit speziellen Komponenten, z. B. zur Erzeugung 
von PDF-Dateien, k\"onnen LaTeX-Publikationen fr die Ver\"offentlichung 
auf CD-ROM oder im Internet vorbereitet werden. Das komplette Satzsystem 
ist frei erh\"altlich und steht praktisch auf allen verbreiteten Betriebssystemen 
zur Verf\"ugung.
LaTeX ist ein m\"achtiges und flexibles Satzsystem, das sich besonders f\"ur
wissenschaftliche und technische Publikationen eignet. Autoren k\"onnen
aus einer Vielzahl von fertigen Layouts ausw\"ahlen und diese eigenen 
Vorstellungen anpassen. Mit speziellen Komponenten, z. B. zur Erzeugung 
von PDF-Dateien, k\"onnen LaTeX-Publikationen fr die Ver\"offentlichung 
auf CD-ROM oder im Internet vorbereitet werden. Das komplette Satzsystem 
ist frei erh\"altlich und steht praktisch auf allen verbreiteten Betriebssystemen 
zur Verf\"ugung.
LaTeX ist ein m\"achtiges und flexibles Satzsystem, das sich besonders f\"ur
wissenschaftliche und technische Publikationen eignet. Autoren k\"onnen
aus einer Vielzahl von fertigen Layouts ausw\"ahlen und diese eigenen 
Vorstellungen anpassen. Mit speziellen Komponenten, z. B. zur Erzeugung 
von PDF-Dateien, k\"onnen LaTeX-Publikationen fr die Ver\"offentlichung 
auf CD-ROM oder im Internet vorbereitet werden. Das komplette Satzsystem 
ist frei erh\"altlich und steht praktisch auf allen verbreiteten Betriebssystemen 
zur Verf\"ugung.

LaTeX ist ein m\"achtiges und flexibles Satzsystem, das sich besonders f\"ur
wissenschaftliche und technische Publikationen eignet. Autoren k\"onnen
aus einer Vielzahl von fertigen Layouts ausw\"ahlen und diese eigenen 
Vorstellungen anpassen. Mit speziellen Komponenten, z. B. zur Erzeugung 
von PDF-Dateien, k\"onnen LaTeX-Publikationen fr die Ver\"offentlichung 
auf CD-ROM oder im Internet vorbereitet werden. Das komplette Satzsystem 
ist frei erh\"altlich und steht praktisch auf allen verbreiteten Betriebssystemen 
zur Verf\"ugung. 

\section{Abschnitt2}
Und dies ist nur der zweite Abschnitt. LaTeX ist ein m\"achtiges und flexibles Satzsystem, das sich besonders f\"ur
wissenschaftliche und technische Publikationen eignet. Autoren k\"onnen
aus einer Vielzahl von fertigen Layouts ausw\"ahlen und diese eigenen 
Vorstellungen anpassen. Mit speziellen Komponenten, z. B. zur Erzeugung 
von PDF-Dateien, k\"onnen LaTeX-Publikationen fr die Ver\"offentlichung 
auf CD-ROM oder im Internet vorbereitet werden. Das komplette Satzsystem 
ist frei erh\"altlich und steht praktisch auf allen verbreiteten Betriebssystemen 
zur Verf\"ugung.

\chapter{Zweites Kapitel}
Dieses Kapitel enth"alt keine weitere Untergliederung.
LaTeX ist ein m\"achtiges und flexibles Satzsystem, das sich besonders f\"ur
wissenschaftliche und technische Publikationen eignet. Autoren k\"onnen
aus einer Vielzahl von fertigen Layouts ausw\"ahlen und diese eigenen 
Vorstellungen anpassen. Mit speziellen Komponenten, z. B. zur Erzeugung 
von PDF-Dateien, k\"onnen LaTeX-Publikationen fr die Ver\"offentlichung 
auf CD-ROM oder im Internet vorbereitet werden. Das komplette Satzsystem 
ist frei erh\"altlich und steht praktisch auf allen verbreiteten Betriebssystemen 
zur Verf\"ugung.

\end{document}
% Hier ist das Dokument nun zu Ende.
